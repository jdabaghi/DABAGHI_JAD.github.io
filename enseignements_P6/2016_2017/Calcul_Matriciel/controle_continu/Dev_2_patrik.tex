\documentclass[11pt]{article}
\usepackage[latin1]{inputenc}
\usepackage[T1]{fontenc}
\usepackage[%
      a4paper,%
      textwidth=18cm,%
      top=2cm,%
      bottom=2cm,%
      headheight=25pt,%
      headsep=12pt,%
      footskip=25pt]{geometry}%
\usepackage[frenchb]{babel}
\parindent0pt
\usepackage{longtable}
\usepackage{amsmath, amsfonts}
\usepackage{enumitem}

\renewcommand{\geq}{\geqslant}
\renewcommand{\leq}{\leqslant}
\def\N{\textrm{I\kern-0.21emN}}
\def\C{\mathbb{C}}
\def\K{\mathbb{K}}
\def\R{\textrm{I\kern-0.21emR}}
\def\Q{\textrm{l\kern-0.5emQ}}
\def\Z{\mathbb{Z}}
\newcommand{\dps}{\displaystyle}

\pagestyle{empty}

\begin{document}

{\Large UPMC \hfill 1M004 - Calcul matriciel \hfill 2016-2017}

\vskip -3mm
\noindent \textbf{\hrulefill}

\vskip 1mm

\center{
{\Large  \textbf{ Section 21.2 -- Contr�le du 24 avril}} \linebreak
{\large \textbf{Dur�e : 1h }}
}

\vskip -2mm
\noindent \textbf{\hrulefill}



%\vskip 2mm

%\centerline{\textbf{Tout appareil �lectronique (calculatrices, t�l�phones portables, etc.) est interdit}}

%\vskip 2mm
{\LARGE
%\begin{center}
%%\begin{tabular}{|p{7cm}|p{1cm}|p{1cm}|p{1cm}|p{1cm}|p{1cm}|p{1cm}|p{1cm}|}
%%\hline
%%\textbf{Num�ro d'�tudiant} &     & & & & &   &
%%\\
%%\hline
%%\end{tabular}
%
%\vskip 0.25cm
%
%\begin{tabular}{|p{2cm}|p{6cm}|p{3cm}|p{5cm}|}
%\hline
%\textbf{Nom} &     & \textbf{Pr�nom   }  &   \\
%\hline
%\end{tabular}
%\end{center}
} %% fin LARGE


{\sl  La pr�sentation et la clart� des raisonnements seront pris en compte dans l'appr�ciation des
copies. Pensez � justifier tous vos r�sultats. }


\vskip 3mm
\hrule

\vskip 3mm
\noindent \textbf{Questions du cours :} \\
\begin{enumerate}[font=\bfseries,label=(Q\arabic*)]
\item  Donner la d�finition d'\textit{une application lin�aire} $F: \R^2 \to \R^2$.
\item  Donner la d�finition d'une application lin�aire $P: \R^3 \to \R^3$ qui est \textit{une projection} dans $\mathbb{R}^3$.

\end{enumerate}

\vskip 0.2cm
  \hrule
\vskip 2mm

\textbf{Exercice 1:}
% \noindent \textbf{Exercices :}\\
 \begin{enumerate}[font=\bfseries]
 \item Soient $A = \begin{pmatrix} 1 & 2 \\ 2 & -1 \end{pmatrix}, \quad B = \begin{pmatrix} 3 & -2 \\ 1 &  1 \end{pmatrix}$. Calculer $\mathrm{det}(A)$, $A^{-1}$, $\mathrm{det}(B)$, $\mathrm{det}(A+B)$, $\mathrm{det}(AB)$, $Tr(AB)$ et $Tr(A+B)$.\\
 Que remarquez-vous? 
 
 \end{enumerate}
 \textbf{Exercice 2:}
 \begin{enumerate}
 \item
D�terminer si le syst�me suivant peut admettre une unique solution
 $
 \begin{cases}
 2x - y + z = 2\\
 3x + y + z = 6\\
 4x + y - 2z = 11
 \end{cases}
 $
 \item
 R�soudre le syst�me par la m�thode d'�limination de Gauss
 \item Soient $$M = \begin{pmatrix} 1 & 0 & 0  \\ 1 & 0 & 1 \\ 2 & 2 & 2 \end{pmatrix}. $$ Montrer que matrice $M$ est inversible. Calculer $M^{-1}$. 
 \end{enumerate}
 \textbf{Exercice 3:}
 \begin{enumerate}[font=\bfseries]
 %----------------------------------------------------- 
 \item Parmi les applications suivantes lesquelles sont lin�aires? (\textit{Justifiez votre r�ponse!}) Lorsque c'est possible, donner la matrice de l'application (dans les bases canoniques).
 \begin{enumerate}
 \item $f: \R^2 \to \R^2; \begin{pmatrix} x \\ y \end{pmatrix} \mapsto \begin{pmatrix} 3x\\ -7y + 5x\end{pmatrix}$
 \item $f: \R^3 \to \R; \begin{pmatrix} x \\ y\\z  \end{pmatrix} \mapsto \sin{(xyz)}$
 \end{enumerate}
 \item
 Soit 
$ A=
  \begin{pmatrix}
  2 & -1 & -1\\
  1 & 0 & -1\\
  1 & -1 & 0
  \end{pmatrix}
 $
 et $g$ l'application lin�aire de $\mathbb{R}^3$ dans $\mathbb{R}^3$ dont la matrice associ�e dans la base canonique de $\mathbb{R}^3$ est $g$.  
\\ $g$ est elle une isom�trie, une projection, une homoth�tie?\\
\begin{center}
\textbf{Exercice 4:}
\end{center} 
On consid�re l'application suivante:
 $f : \R^3 \longrightarrow \R^3$, $\begin{pmatrix} x \\ y\\z  \end{pmatrix} \longmapsto \begin{pmatrix} y+z \\ x+y+z\\x  \end{pmatrix}$
 
 \begin{enumerate}
 \item Montrez que $f$ est lin�aire.
 \item Donner l'expression de $F$, la matrice de $f$ dans la base canonique de $\R^3$.
 \item D�terminer l'ensemble ker($f$)$= \lbrace x \in \R^3\,|\, f(x) = 0 \rbrace$
 \item D�terminer l'ensemble im($f$)$= \lbrace f(x)\,,\,x \in \R^3\ \rbrace$
 \end{enumerate}
%------------------------------------------------------
 
 \end{enumerate}

\end{document}
